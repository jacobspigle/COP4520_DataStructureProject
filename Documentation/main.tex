%%%%%%%%%%%%%%%%%%%%%%%%%%%%%%%%%%%%%%%%%%%%%%%%%%%%%%%%%%%%%%%%%%%%%%%%%%%%%%%%
%2345678901234567890123456789012345678901234567890123456789012345678901234567890
%        1         2         3         4         5         6         7         8

\documentclass[letterpaper, 10 pt, conference]{ieeeconf}  % Comment this line out
% if you need a4paper
%\documentclass[a4paper, 10pt, conference]{ieeeconf}      % Use this line for a4
% paper

\IEEEoverridecommandlockouts                              % This command is only
% needed if you want to
% use the \thanks command
\overrideIEEEmargins
% See the \addtolength command later in the file to balance the column lengths
% on the last page of the document



% The following packages can be found on http:\\www.ctan.org
%\usepackage{graphics} % for pdf, bitmapped graphics files
%\usepackage{epsfig} % for postscript graphics files
%\usepackage{mathptmx} % assumes new font selection scheme installed
%\usepackage{times} % assumes new font selection scheme installed
%\usepackage{amsmath} % assumes amsmath package installed
%\usepackage{amssymb}  % assumes amsmath package installed

\title{\LARGE \bf
	Concurrent Wait-Free Red Black Trees
}

%\author{ \parbox{3 in}{\centering Huibert Kwakernaak*
%         \thanks{*Use the $\backslash$thanks command to put information here}\\
%         Faculty of Electrical Engineering, Mathematics and Computer Science\\
%         University of Twente\\
%         7500 AE Enschede, The Netherlands\\
%         {\tt\small h.kwakernaak@autsubmit.com}}
%         \hspace*{ 0.5 in}
%         \parbox{3 in}{ \centering Pradeep Misra**
%         \thanks{**The footnote marks may be inserted manually}\\
%        Department of Electrical Engineering \\
%         Wright State University\\
%         Dayton, OH 45435, USA\\
%         {\tt\small pmisra@cs.wright.edu}}
%}

\author{David Ferguson and Jacob Spigle% <-this % stops a space
	\thanks{$^{1}$D. Ferguson is a student at Department of Electrical Engineering and Computer Science, University of Central Florida, Orlando, Florida, 32816-2450 }%
	\thanks{$^{2}$J. Spigle is a student at Department of Electrical Engineering and Computer Science, University of Central Florida, Orlando, Florida, 32816-2450 %
}}

\begin{document}
	
	
	
	\maketitle
	\thispagestyle{empty}
	\pagestyle{empty}
	
	
	%%%%%%%%%%%%%%%%%%%%%%%%%%%%%%%%%%%%%%%%%%%%%%%%%%%%%%%%%%%%%%%%%%%%%%%%%%%%%%%%
	\begin{abstract}
		
		Our re-implementation of Natarajan, Savoie, and Mittal's wait-free algorithm \cite{c1} seeks to not only imitate the concurrent data structure presented, but also transform it into a transactional data structure using the RSTM library. We are implementing a concurrently managed red-black tree using wait-free algorithms designed and presented in \cite{RedBlackDoc}. We will be writing these algorithms using C++ as our programming language.  Concurrency during manipulation of a tree data structure is not plausible without additional (and creative) data structures because of the multiple instructions that rotations perform during the re-balancing of the tree. Using the techniques outlined in \cite{RedBlackDoc}, we will have a concurrent algorithm that makes progress, is linearizable, and correct. Experiments run by the authors of this implementation prove that their solution provides ``significantly better performance''\cite{RedBlackDoc} than other attempts that preceded it, including both attempts at concurrency and lock-based implementations.
		
	\end{abstract}
	
	
	\section{Introduction}
	This wait-free implementation of the Red Black tree data structure boasts search(), insert(), update(), and delete() functions, all executed utilizing single-word compare-and-swap instructions. The data structure's concurrent implementation employs the use of ``windows'', which are overlapping snapshots of the current state of the Red Black tree within the scope of the windows' root node. Each of these windows is a balanced Red Black tree itself, and pushing a modified window into the windows' origin will result in a correct, linearizable solution. This is because the window itself can be atomically swapped, where rotations are done inside a modified window, and using a single-word compare-and-swap, are placed back into the node where the window originated. This solution also strives for optimal concurrency by introducing an array that holds pending instructions (using the \textit{announce} variable) and decides whether or not a thread will assist by checking for conflicts with it's own update operation (using a \textit{gate} variable, given to each record in the tree). An modify operation may also help during a search operation to ensure that the search operation eventually terminates \cite{RedBlackDoc}. This is necessary because this implementation avoids copying windows unnecessarily, and instead traversing to the next root when such a transaction would occur. These additions to the traditional sequential Red Black tree allow for an efficient algorithm that has outperformed other attempts at this implementation of the concurrent wait-free Red Black tree data structure. \\ \\
	\textit{Related Work: } There have been a few more recent attempts at this wait-free implementation of manipulating red-black trees concurrently. Notably, in 2014, there was a thesis written proposing that instead of a Top-Down approach to obtaining ownership of nodes within the tree, that working from Bottom-Up approach would ``[allow] operations interested in completely disparate portions of the tree to execute entirely uninhibited''\cite{BottomUpRBD}.
	
	\section{Current State, Problems, and Planning}
	We do not foresee many changes between the given implementation in \cite{RedBlackDoc} and our own, because the psuedo code written in the document is easily translatable to actual C++ code. However, our own implementation's class structure will diverge from their own, because we will be reverse engineering it from references in the algorithms provided.
	\par Because we are developing in a multithreaded fashion in C++, we are using the POSIX Threads library (pthread.h) to create, join, and monitor our threads. For atomic operations such as Compare and Swap (CAS), and for utilization of atomic variables we will be using the C++ Standard Library (std::atomic). The usage of these libraries will be the tools we use to create the concurrent data structure.
	\par As of right now, our task is to properly be able to handle the external record for a node's value, as it is not being stored inside the node itself for this implementation. We have encountered some issues simply with translation of these algorithms, figuring out the best way to bring them into the code without sacrificing correctness. We struggled a bit with deciding whether keeping the decision made by the authors of this implementation to ``assume the tree is never empty and always contains at least one node''\cite{RedBlackDoc}. For the purpose of getting our solution up and running, we are current following the practices of the authors.
	\par Our plan of action for completing this project is to have a working solution for the concurrent data structure by the end of March. Moving forward onto part two from April till the final due date. In the event that we are able to complete both parts earlier than expected, we hope to conduct further testing and experimentation, possibly further improving upon our solution or rectifying any overlooked sections where our implementation is unoptimized.
	\section{PROCEDURE FOR PAPER SUBMISSION}
	
	\subsection{Selecting a Template (Heading 2)}
	
	First, confirm that you have the correct template for your paper size. This template has been tailored for output on the US-letter paper size. Please do not use it for A4 paper since the margin requirements for A4 papers may be different from Letter paper size.
	
	\subsection{Maintaining the Integrity of the Specifications}
	
	The template is used to format your paper and style the text. All margins, column widths, line spaces, and text fonts are prescribed; please do not alter them. You may note peculiarities. For example, the head margin in this template measures proportionately more than is customary. This measurement and others are deliberate, using specifications that anticipate your paper as one part of the entire proceedings, and not as an independent document. Please do not revise any of the current designations
	
	\section{MATH}
	
	Before you begin to format your paper, first write and save the content as a separate text file. Keep your text and graphic files separate until after the text has been formatted and styled. Do not use hard tabs, and limit use of hard returns to only one return at the end of a paragraph. Do not add any kind of pagination anywhere in the paper. Do not number text heads-the template will do that for you.
	
	Finally, complete content and organizational editing before formatting. Please take note of the following items when proofreading spelling and grammar:
	
	\subsection{Abbreviations and Acronyms} Define abbreviations and acronyms the first time they are used in the text, even after they have been defined in the abstract. Abbreviations such as IEEE, SI, MKS, CGS, sc, dc, and rms do not have to be defined. Do not use abbreviations in the title or heads unless they are unavoidable.
	
	\subsection{Units}
	
	\begin{itemize}
		
		\item Use either SI (MKS) or CGS as primary units. (SI units are encouraged.) English units may be used as secondary units (in parentheses). An exception would be the use of English units as identifiers in trade, such as Ò3.5-inch disk drive.
		\item Avoid combining SI and CGS units, such as current in amperes and magnetic field in oersteds. This often leads to confusion because equations do not balance dimensionally. If you must use mixed units, clearly state the units for each quantity that you use in an equation.
		\item Do not mix complete spellings and abbreviations of units: ÒWb/m2 or Òwebers per square meter, not Òwebers/m2.  Spell out units when they appear in text: Ò. . . a few henries, not Ò. . . a few H.
		\item Use a zero before decimal points: Ò0.25, not Ò.25. Use Òcm3, not Òcc. (bullet list)
		
	\end{itemize}
	
	
	\subsection{Equations}
	
	The equations are an exception to the prescribed specifications of this template. You will need to determine whether or not your equation should be typed using either the Times New Roman or the Symbol font (please no other font). To create multileveled equations, it may be necessary to treat the equation as a graphic and insert it into the text after your paper is styled. Number equations consecutively. Equation numbers, within parentheses, are to position flush right, as in (1), using a right tab stop. To make your equations more compact, you may use the solidus ( / ), the exp function, or appropriate exponents. Italicize Roman symbols for quantities and variables, but not Greek symbols. Use a long dash rather than a hyphen for a minus sign. Punctuate equations with commas or periods when they are part of a sentence, as in
	
	$$
	\alpha + \beta = \chi \eqno{(1)}
	$$
	
	Note that the equation is centered using a center tab stop. Be sure that the symbols in your equation have been defined before or immediately following the equation. Use Ò(1), not ÒEq. (1) or Òequation (1), except at the beginning of a sentence: ÒEquation (1) is . . .
	
	\subsection{Some Common Mistakes}
	\begin{itemize}
		
		
		\item The word Òdata is plural, not singular.
		\item The subscript for the permeability of vacuum ?0, and other common scientific constants, is zero with subscript formatting, not a lowercase letter Òo.
		\item In American English, commas, semi-/colons, periods, question and exclamation marks are located within quotation marks only when a complete thought or name is cited, such as a title or full quotation. When quotation marks are used, instead of a bold or italic typeface, to highlight a word or phrase, punctuation should appear outside of the quotation marks. A parenthetical phrase or statement at the end of a sentence is punctuated outside of the closing parenthesis (like this). (A parenthetical sentence is punctuated within the parentheses.)
		\item A graph within a graph is an Òinset, not an Òinsert. The word alternatively is preferred to the word Òalternately (unless you really mean something that alternates).
		\item Do not use the word Òessentially to mean Òapproximately or Òeffectively.
		\item In your paper title, if the words Òthat uses can accurately replace the word Òusing, capitalize the Òu; if not, keep using lower-cased.
		\item Be aware of the different meanings of the homophones Òaffect and Òeffect, Òcomplement and Òcompliment, Òdiscreet and Òdiscrete, Òprincipal and Òprinciple.
		\item Do not confuse Òimply and Òinfer.
		\item The prefix Ònon is not a word; it should be joined to the word it modifies, usually without a hyphen.
		\item There is no period after the Òet in the Latin abbreviation Òet al..
		\item The abbreviation Òi.e. means Òthat is, and the abbreviation Òe.g. means Òfor example.
		
	\end{itemize}
	
	
	\section{USING THE TEMPLATE}
	
	Use this sample document as your LaTeX source file to create your document. Save this file as {\bf root.tex}. You have to make sure to use the cls file that came with this distribution. If you use a different style file, you cannot expect to get required margins. Note also that when you are creating your out PDF file, the source file is only part of the equation. {\it Your \TeX\ $\rightarrow$ PDF filter determines the output file size. Even if you make all the specifications to output a letter file in the source - if you filter is set to produce A4, you will only get A4 output. }
	
	It is impossible to account for all possible situation, one would encounter using \TeX. If you are using multiple \TeX\ files you must make sure that the ``MAIN`` source file is called root.tex - this is particularly important if your conference is using PaperPlaza's built in \TeX\ to PDF conversion tool.
	
	\subsection{Headings, etc}
	
	Text heads organize the topics on a relational, hierarchical basis. For example, the paper title is the primary text head because all subsequent material relates and elaborates on this one topic. If there are two or more sub-topics, the next level head (uppercase Roman numerals) should be used and, conversely, if there are not at least two sub-topics, then no subheads should be introduced. Styles named ÒHeading 1, ÒHeading 2, ÒHeading 3, and ÒHeading 4 are prescribed.
	
	\subsection{Figures and Tables}
	
	Positioning Figures and Tables: Place figures and tables at the top and bottom of columns. Avoid placing them in the middle of columns. Large figures and tables may span across both columns. Figure captions should be below the figures; table heads should appear above the tables. Insert figures and tables after they are cited in the text. Use the abbreviation ÒFig. 1, even at the beginning of a sentence.
	
	\begin{table}[h]
		\caption{An Example of a Table}
		\label{table_example}
		\begin{center}
			\begin{tabular}{|c||c|}
				\hline
				One & Two\\
				\hline
				Three & Four\\
				\hline
			\end{tabular}
		\end{center}
	\end{table}
	
	
	\begin{figure}[thpb]
		\centering
		\framebox{\parbox{3in}{We suggest that you use a text box to insert a graphic (which is ideally a 300 dpi TIFF or EPS file, with all fonts embedded) because, in an document, this method is somewhat more stable than directly inserting a picture.
		}}
		%\includegraphics[scale=1.0]{figurefile}
		\caption{Inductance of oscillation winding on amorphous
			magnetic core versus DC bias magnetic field}
		\label{figurelabel}
	\end{figure}
	
	
	Figure Labels: Use 8 point Times New Roman for Figure labels. Use words rather than symbols or abbreviations when writing Figure axis labels to avoid confusing the reader. As an example, write the quantity ÒMagnetization, or ÒMagnetization, M, not just ÒM. If including units in the label, present them within parentheses. Do not label axes only with units. In the example, write ÒMagnetization (A/m) or ÒMagnetization {A[m(1)]}, not just ÒA/m. Do not label axes with a ratio of quantities and units. For example, write ÒTemperature (K), not ÒTemperature/K.
	
	\section{CONCLUSIONS}
	
	A conclusion section is not required. Although a conclusion may review the main points of the paper, do not replicate the abstract as the conclusion. A conclusion might elaborate on the importance of the work or suggest applications and extensions. 
	
	\addtolength{\textheight}{-12cm}   % This command serves to balance the column lengths
	% on the last page of the document manually. It shortens
	% the textheight of the last page by a suitable amount.
	% This command does not take effect until the next page
	% so it should come on the page before the last. Make
	% sure that you do not shorten the textheight too much.
	
	%%%%%%%%%%%%%%%%%%%%%%%%%%%%%%%%%%%%%%%%%%%%%%%%%%%%%%%%%%%%%%%%%%%%%%%%%%%%%%%%
	
	
	
	%%%%%%%%%%%%%%%%%%%%%%%%%%%%%%%%%%%%%%%%%%%%%%%%%%%%%%%%%%%%%%%%%%%%%%%%%%%%%%%%
	
	
	
	%%%%%%%%%%%%%%%%%%%%%%%%%%%%%%%%%%%%%%%%%%%%%%%%%%%%%%%%%%%%%%%%%%%%%%%%%%%%%%%%
	\section*{APPENDIX}
	
	Appendixes should appear before the acknowledgment.
	
	\section*{ACKNOWLEDGMENT}
	
	The preferred spelling of the word Òacknowledgment in America is without an e after the g. Avoid the stilted expression, ÒOne of us (R. B. G.) thanks . . .  Instead, try ÒR. B. G. thanks. Put sponsor acknowledgments in the unnumbered footnote on the first page.
	
	
	
	%%%%%%%%%%%%%%%%%%%%%%%%%%%%%%%%%%%%%%%%%%%%%%%%%%%%%%%%%%%%%%%%%%%%%%%%%%%%%%%%
	
	References are important to the reader; therefore, each citation must be complete and correct. If at all possible, references should be commonly available publications.
	
	
	
	\begin{thebibliography}{99}
		
		%@manual{ RedBlackDoc,
		%	title = {Concurrent Wait-Free Red Black Trees},
		%	author = {Aravind Natarajan and Lee H. Savoie and Neeraj Mittal},
		%	organization = {The University of Texas at Dallas},
		%	address = {Richardson, TX 75080, USA},
		%	year = {2013},}
		
		%@manual{ BottomUpRBD,
		%	title = {Concurrent Localized Wait-Free Operations on a Red Black Tree},
		%	author = {Vitaliy Kubushyn},
		%	organization = {University of Nevada, Las Vegas},
		%	year = {2014},}
		
		\bibitem{c1} A. Natarajan, L. Savoie, \& N. Mittal 2013, 'Concurrent Wait-Free Red Black Trees', The University of Texas at Dallas, Richardson, TX 75080, USA
		\bibitem{c2} V. Kubushyn 2014, 'Concurrent Localized Wait-Free Operations on a Red Black Tree', University of Nevada, Las Vegas
		\bibitem{c3} H. Poor, An Introduction to Signal Detection and Estimation.   New York: Springer-Verlag, 1985, ch. 4.
		\bibitem{c4} B. Smith, ÒAn approach to graphs of linear forms (Unpublished work style), unpublished.
		\bibitem{c5} E. H. Miller, ÒA note on reflector arrays (Periodical styleÑAccepted for publication), IEEE Trans. Antennas Propagat., to be publised.
		\bibitem{c6} J. Wang, ÒFundamentals of erbium-doped fiber amplifiers arrays (Periodical styleÑSubmitted for publication), IEEE J. Quantum Electron., submitted for publication.
		\bibitem{c7} C. J. Kaufman, Rocky Mountain Research Lab., Boulder, CO, private communication, May 1995.
		\bibitem{c8} Y. Yorozu, M. Hirano, K. Oka, and Y. Tagawa, ÒElectron spectroscopy studies on magneto-optical media and plastic substrate interfaces(Translation Journals style), IEEE Transl. J. Magn.Jpn., vol. 2, Aug. 1987, pp. 740Ð741 [Dig. 9th Annu. Conf. Magnetics Japan, 1982, p. 301].
		\bibitem{c9} M. Young, The Techincal Writers Handbook.  Mill Valley, CA: University Science, 1989.
		\bibitem{c10} J. U. Duncombe, ÒInfrared navigationÑPart I: An assessment of feasibility (Periodical style), IEEE Trans. Electron Devices, vol. ED-11, pp. 34Ð39, Jan. 1959.
		\bibitem{c11} S. Chen, B. Mulgrew, and P. M. Grant, ÒA clustering technique for digital communications channel equalization using radial basis function networks, IEEE Trans. Neural Networks, vol. 4, pp. 570Ð578, July 1993.
		\bibitem{c12} R. W. Lucky, ÒAutomatic equalization for digital communication, Bell Syst. Tech. J., vol. 44, no. 4, pp. 547Ð588, Apr. 1965.
		\bibitem{c13} S. P. Bingulac, ÒOn the compatibility of adaptive controllers (Published Conference Proceedings style), in Proc. 4th Annu. Allerton Conf. Circuits and Systems Theory, New York, 1994, pp. 8Ð16.
		\bibitem{c14} G. R. Faulhaber, ÒDesign of service systems with priority reservation, in Conf. Rec. 1995 IEEE Int. Conf. Communications, pp. 3Ð8.
		\bibitem{c15} W. D. Doyle, ÒMagnetization reversal in films with biaxial anisotropy, in 1987 Proc. INTERMAG Conf., pp. 2.2-1Ð2.2-6.
		\bibitem{c16} G. W. Juette and L. E. Zeffanella, ÒRadio noise currents n short sections on bundle conductors (Presented Conference Paper style), presented at the IEEE Summer power Meeting, Dallas, TX, June 22Ð27, 1990, Paper 90 SM 690-0 PWRS.
		\bibitem{c17} J. G. Kreifeldt, ÒAn analysis of surface-detected EMG as an amplitude-modulated noise, presented at the 1989 Int. Conf. Medicine and Biological Engineering, Chicago, IL.
		\bibitem{c18} J. Williams, ÒNarrow-band analyzer (Thesis or Dissertation style), Ph.D. dissertation, Dept. Elect. Eng., Harvard Univ., Cambridge, MA, 1993. 
		\bibitem{c19} N. Kawasaki, ÒParametric study of thermal and chemical nonequilibrium nozzle flow, M.S. thesis, Dept. Electron. Eng., Osaka Univ., Osaka, Japan, 1993.
		\bibitem{c20} J. P. Wilkinson, ÒNonlinear resonant circuit devices (Patent style), U.S. Patent 3 624 12, July 16, 1990. 
		
		
		
		
		
		
	\end{thebibliography}
	
	
	
	
\end{document}

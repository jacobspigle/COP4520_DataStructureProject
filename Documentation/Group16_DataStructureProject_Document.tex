\documentclass[12pt,a4paper]{article}
\usepackage{amsmath}
\usepackage{amsfonts}
\usepackage{amssymb}
\usepackage{graphicx}
\usepackage{multicol}
\usepackage[utf8]{inputenc}
\usepackage{array}
\usepackage{lipsum}
\graphicspath{{images/}}
\usepackage{parskip}
\usepackage{indentfirst}
\usepackage{courier}
\usepackage{titlesec}
\usepackage{wrapfig}
\usepackage[numbers]{natbib}
\usepackage{caption}
\usepackage[left=1.5in, right=1in, top = 1in, bottom = 1in]{geometry}
\usepackage{float}
\usepackage[ruled]{algorithm}
\usepackage{algpseudocode}
\usepackage{url}
\usepackage{rotating}
\geometry{headheight = 12pt}
\parindent 15pt
\parskip 2ex

\usepackage{filecontents}
\begin{filecontents}{citations.bib}
	{
		@manual{ RedBlackDoc,
			title = {Concurrent Wait-Free Red Black Trees},
			author = {Aravind Natarajan and Lee H. Savoie and Neeraj Mittal},
			organization = {The University of Texas at Dallas},
			address = {Richardson, TX 75080, USA},
			year = {2013},}
		
		%add references here
\end{filecontents}

\newcommand{\WRP}{\par\qquad\(\hookrightarrow\)\enspace}

\renewcommand{\listfigurename}{Figures}
\renewcommand{\listtablename}{Tables}

\titleformat{\chapter}[display]   
{\normalfont\huge\bfseries}{\chaptertitlename\ \thechapter}{20pt}{\Huge}   
\titlespacing*{\chapter}{0pt}{-50pt}{20pt}

\author{Group 16 \\ -------------------- \\David Ferguson, Jacob Spigle}
\title{Concurrent Wait-Free Red Black Trees}

\begin{document}
	
\pagenumbering{Roman} 
\maketitle
\begin{abstract}
	This wait-free implementation of the Red Black tree data structure boasts search(), insert(), update(), and delete() functions, all executed utilizing single-word compare-and-swap instructions. The data structure's concurrent implementation employs the use of ``windows'', overlapping snapshots of the current state of the Red Black tree within the scope of the windows' root node. Each of these windows is a balanced Red Black tree itself, and pushing a modified window into the windows' origin will result in a correct, linearizable solution. \par This solution also strives for optimal concurrency by introducing an array that holds pending instructions (\textit{announce}) and decides whether or not a thread will assist by checking for conflicts with it's own update operation (using \textit{gate}). \par An modify operation may also help during a search operation to ensure that the search operation eventually terminates \cite{RedBlackDoc}. This is necessary because this implementation avoids copying windows unnecessarily, and instead traversing to the next root when such a transaction would occur. \par These additions to the traditional sequential Red Black tree allow for an efficient algorithm that has outperformed other attempts at this implementation of the concurrent wait-free Red Black tree data structure.
\end{abstract}
\newpage
\pagenumbering{arabic}
\section{Introduction}



\newpage
\addcontentsline{toc}{section}{Bibliography}
\bibliographystyle{unsrt}
\bibliography{citations}

\end{document}